\chapter{Requisitos}
\section{Escenarios de caso de uso}
\begin{table}[h]
	\centering
	\begin{tabular}{|lp{0.8\textwidth}|}
		\hline
		\rowcolor[HTML]{C0C0C0} 
		\textbf{CU-01}   & \textbf{Nombre de escenario} \\ \hline
		Participantes    &                              \\ \hline
		Flujo de eventos &                              \\ \hline
		Importancia      &                              \\ \hline
		Versión          &                              \\ \hline
	\end{tabular}
	\caption{CU-01: Ejemplo}
\end{table}

\begin{table}[H]
	\centering
	\begin{tabular}{|lp{0.8\textwidth}|}
		\hline
		\rowcolor[HTML]{C0C0C0} 
		\textbf{CU-02}   & \textbf{MuerteJugador} \\ \hline
		Participantes    & \begin{itemize} 
								\item \textbf{Jugador: PlayerEntity}
								\item \textbf{Daño: DamageEntity}
								\item \textbf{Salud: PlayerHP}
								\item \textbf{Escudo: PlayerShield}
							\end{itemize}               \\ \hline
		Flujo de eventos &  \begin{enumerate}
								\item \textbf{Jugador} colisiona con \textbf{DamageEntity}
								\item Valor de \textbf{Escudo} se reduce de acuerdo al valor de \textbf{DamageEntity}
								\item Valor de \textbf{PlayerHP} se reduce de acuerdo al valor de \textbf{DamageEntity} y \textbf{Shield}.
								\begin{enumerate}
									\item \textbf{PlayerHP} está por encima de 0, continuar con normalidad.
									\item \textbf{PlayerHP} está por debajo de 0 o es igual, \textbf{Jugador} muere.
									\begin{enumerate}
										\item \textbf{Jugador} desaparece.
										\item Aparece una animación de explosión en la posición del \textbf{Jugador}.
										\item Al finalizar el juego, se le pide al \textbf{Jugador} si quiere volver a jugar o quiere volver al menú principal.
									\end{enumerate}
								\end{enumerate}
							\end{enumerate}                            \\ \hline
		Importancia      &  Alta                            \\ \hline
		Versión          &  1.0.0                            \\ \hline
	\end{tabular}
	\caption{CU-02: MuerteJugador}
\end{table}
\section{Diagramas de secuencia}
\begin{center}
	\fbox{\begin{minipage}{32em}
			\textit{Para cada escenario anterior, poner aquí un diagrama de secuencia.}
	\end{minipage}}
\end{center}
\section{Diagrama de clases}
\begin{center}
	\fbox{\begin{minipage}{32em}
			\textit{Poner aquí el diagrama de clases de todo el juego donde se pueda ver las relaciones entre ellas. Explicar brevemente el diagrama.}
	\end{minipage}}
\end{center}