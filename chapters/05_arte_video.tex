\chapter{Arte y vídeo}
\section{Objetivos}
\begin{center}
	\fbox{\begin{minipage}{32em}
			\textit{¿Qué se espera conseguir con el estilo del arte elegido? Por ejemplo, ¿se elige modelar en low poly para reducir presupuesto o meramente por cuestiones estéticas ya que esta muy relacionado con la temática del juego? }
	\end{minipage}}
\end{center}
\section{Arte y animación 2D}
\subsection{GUI}
\begin{center}
	\fbox{\begin{minipage}{32em}
			\textit{Ventanas, punteros, marcadores, íconos, botones, menús, etc. Todos estos elementos deben recogerse en esta sección, con una descripción de para que se utilizan y una representación gráfica de ellas. }
	\end{minipage}}
\end{center}
\subsection{Entorno}
\begin{center}
	\fbox{\begin{minipage}{32em}
			\textit{Decorados de terreno, texturas, skyboxes, fondos, tiles... Todos estos elementos deben recogerse en esta sección, con una descripción de para que se utilizan y una representación gráfica de ellas. }
	\end{minipage}}
\end{center}
\subsection{Elementos para el gameplay}
\begin{center}
	\fbox{\begin{minipage}{32em}
			\textit{Animaciones de enemigos y jugadores (sprites o modelos), objetos, armas, power-ups, etc. Todos estos elementos deben recogerse en esta sección, con una descripción de para que se utilizan y una representación gráfica de ellas. }
	\end{minipage}}
\end{center}
\section{Arte y animación 3D}
\subsection{GUI}
\begin{center}
	\fbox{\begin{minipage}{32em}
			\textit{Ventanas, punteros, marcadores, íconos, botones, menús, etc. Todos estos elementos deben recogerse en esta sección, con una descripción de para que se utilizan y una representación gráfica de ellas. }
	\end{minipage}}
\end{center}
\subsection{Entorno}
\begin{center}
	\fbox{\begin{minipage}{32em}
			\textit{Decorados de terreno, texturas, skyboxes, fondos, tiles... Todos estos elementos deben recogerse en esta sección, con una descripción de para que se utilizan y una representación gráfica de ellas. }
	\end{minipage}}
\end{center}
\subsection{Elementos para el gameplay}
\begin{center}
	\fbox{\begin{minipage}{32em}
			\textit{Animaciones de enemigos y jugadores (sprites o modelos), objetos, armas, power-ups, etc. Todos estos elementos deben recogerse en esta sección, con una descripción de para que se utilizan y una representación gráfica de ellas. }
	\end{minipage}}
\end{center}
\section{Cinemáticas}
\begin{center}
	\fbox{\begin{minipage}{32em}
			\textit{Listar todas las cinemáticas que hay en el juego y especificar su finalidad, contenido y la duración. }
	\end{minipage}}
\end{center}