\chapter{Gestión}
\section{Equipo de desarrollo}
\begin{center}
	\fbox{\begin{minipage}{32em}
			\textit{Indicar las personas que van a participar en este desarrollo, incluyendo un breve resumen de su CV, el puesto a ocupar en el proyecto (desarrollador, diseñador, project manager, tester...) y las responsabilidades}
	\end{minipage}}
\end{center}
\section{Actividades}
\begin{center}
	\fbox{\begin{minipage}{32em}
			\textit{Descomponer el trabajo en actividades para más adelante facilitar la elaboración del diagrama de Gantt. Cada actividad tendrá asociado un identificador único, un título descriptivo, una fecha de inicio y una fecha de fin y las personas que se van a encargar de llevar a cabo la actividad. }
	\end{minipage}}
\end{center}
\section{Hitos}
\begin{center}
	\fbox{\begin{minipage}{32em}
			\textit{Identificar los hitos del proyecto a medida que se van realizando las actividades. Cada hito tendrá asociado una fecha, un nombre y una descripción. }
	\end{minipage}}
\end{center}
\section{Diagrama de Gantt}
\begin{center}
	\fbox{\begin{minipage}{32em}
			\textit{Poner un diagrama de Gantt con la información anteriormente especificada. }
	\end{minipage}}
\end{center}
\section{Análisis de coste}
\begin{center}
	\fbox{\begin{minipage}{32em}
			\textit{Analizar el coste que tendría desarrollar este proyecto desde principio a fin, contando con gastos hardware y software, como de personal y otro tipo de gastos (alquiler, facturas de luz, Internet, etc.) }
	\end{minipage}}
\end{center}
\section{Análisis de riesgos}
\begin{center}
	\fbox{\begin{minipage}{32em}
			\textit{Especificar los posibles riesgos. Esta especificación será a alto nivel ya que todavía no se conocen muchos detalles del proyecto. }
	\end{minipage}}
\end{center}
