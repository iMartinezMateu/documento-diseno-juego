\chapter{Sonido y música}
\section{Objetivos}
\begin{center}
	\fbox{\begin{minipage}{32em}
			\textit{Objetivos técnicos y estéticos para la parte sonora del juego. Describir las emociones que se pretenden inducir al jugador. Nombrar juegos existentes o películas como ejemplos para detallar mejor esta sección. }
	\end{minipage}}
\end{center}
\section{Efectos de sonido}
\subsection{GUI}
\begin{center}
	\fbox{\begin{minipage}{32em}
			\textit{Clicks en botones, apertura de ventanas, etc. Todos estos elementos deben recogerse en esta sección, con una descripción de para que se utilizan. }
	\end{minipage}}
\end{center}
\subsection{Efectos especiales}
\begin{center}
	\fbox{\begin{minipage}{32em}
			\textit{Explosiones, disparos, etc. Todos estos elementos deben recogerse en esta sección, con una descripción de para que se utilizan. }
	\end{minipage}}
\end{center}
\subsection{Personajes}
\begin{center}
	\fbox{\begin{minipage}{32em}
			\textit{Voces, grabaciones de radio, colisiones, etc. Todos estos elementos deben recogerse en esta sección, con una descripción de para que se utilizan. }
	\end{minipage}}
\end{center}
\subsection{Entorno}
\begin{center}
	\fbox{\begin{minipage}{32em}
			\textit{ Ruidos de animales, ruido de lluvia, viento, etc. Todos estos elementos deben recogerse en esta sección, con una descripción de para que se utilizan. }
	\end{minipage}}
\end{center}
\section{Música}
\begin{center}
	\fbox{\begin{minipage}{32em}
			\textit{Listar todas las canciones que se utilizan en el juego, que intentan expresar al jugador, cuando se usan y cuando se deberían reutilizar.}
	\end{minipage}}
\end{center}
