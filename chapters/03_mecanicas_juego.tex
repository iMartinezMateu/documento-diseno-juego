\chapter{Gameplay y mecánicas del juego}
\section{Gameplay}
\subsection{Núcleo}
\begin{center}
	\fbox{\begin{minipage}{32em}
			\textit{En pocos párrafos, detallar la esencia del juego. Estas palabras serán las semillas a partir de las cuales el diseño del juego irá creciendo, ayudando así a que el juego tenga bastante éxito en el mercado. El contenido de esta sección es similar a la del concepto del juego pero expresado de forma más esquemática, como un listado con viñetas.}
	\end{minipage}}
\end{center}
\subsection{Objetivos}
\begin{center}
	\fbox{\begin{minipage}{32em}
			\textit{Detallar los objetivos que el usuario tiene que superar para dar por completado el juego}
	\end{minipage}}
\end{center}
\subsection{Progresión}
\begin{center}
	\fbox{\begin{minipage}{32em}
			\textit{Trazar el flujo típico con una descripción detallada de lo que tiene que hacer el jugador para ir progresando en el juego e ir cumpliendo los objetivos. Si la sección anterior era la raíz de un árbol, esta sección constituye el tronco y las ramas. }
	\end{minipage}}
\end{center}
\section{Mecánicas}
\subsection{Movimiento}
\begin{center}
	\fbox{\begin{minipage}{32em}
			\textit{¿Cómo se deben mover los personajes de acuerdo a los eventos que se sucedan en el juego? Un ejemplo de esto sería, los vehículos reducirán su velocidad cuando estén subiendo una pendiente y aumentarán su velocidad cuando estén bajando una pendiente.}
	\end{minipage}}
\end{center}
\subsection{Objetos}
\begin{center}
	\fbox{\begin{minipage}{32em}
			\textit{Describir la forma de obtención de objetos dentro del juego y la forma de que el jugador pueda usarlos. }
	\end{minipage}}
\end{center}
\subsection{Acciones}
\begin{center}
	\fbox{\begin{minipage}{32em}
			\textit{Describir las acciones que se desencadenan cuando se pulsan botones o interruptores, cuando se interactúa con objetos dentro del propio juego, etc. }
	\end{minipage}}
\end{center}
\subsection{Combate}
\begin{center}
	\fbox{\begin{minipage}{32em}
			\textit{Detallar y modelar el escenario de combate si es que lo hubiera en el juego. }
	\end{minipage}}
\end{center}
\subsection{Economía}
\begin{center}
	\fbox{\begin{minipage}{32em}
			\textit{Detallar el sistema económico que implementa el juego. Esta sección es muy importante para juegos RPG y debe detallarse su funcionamiento aquí.}
	\end{minipage}}
\end{center}
\subsection{Inteligencia Artificial}
\begin{center}
	\fbox{\begin{minipage}{32em}
			\textit{Describir el comportamiento esperado de la inteligencia artificial del juego. Esto incluye el movimiento (path finding), reacciones y disparadores, selección de objetivo y otro tipo de decisiones estratégicas o de combate o de interacción con otros elementos del juego.  }
	\end{minipage}}
\end{center}
\section{Flujo de pantallas}
\begin{center}
	\fbox{\begin{minipage}{32em}
			\textit{Boceto que muestre la relación entre diferentes pantallas del juego. Por ejemplo, la escena A se relaciona con la escena B cuando el jugador lleva a cabo una acción, entonces representar de forma gráfica que es lo que tiene que hacer el jugador para pasar de la escena A a la escena B. Para cada escena, describir de forma detallada su objetivo y utilidad. }
	\end{minipage}}
\end{center}
\section{Rejugabilidad}
\begin{center}
	\fbox{\begin{minipage}{32em}
			\textit{Detallar si el juego ofrece otros desafios al jugador una vez este haya hecho la historia principal y haya cumplido los objetivos principales del juego.}
	\end{minipage}}
\end{center}
\section{Almacenamiento del estado de la partida}
\begin{center}
	\fbox{\begin{minipage}{32em}
			\textit{Detallar que tipo de datos se van a almacenar en un sistema de almacenamiento no volátil cuando el jugador quiera guardar la partida para no perder su progreso.}
	\end{minipage}}
\end{center}
\section{Configuración}
\begin{center}
	\fbox{\begin{minipage}{32em}
			\textit{Detallar si el juego ofrece la posibilidad al jugador de configurar ciertos aspectos del juego. Si esto es afirmativo, explicar qué es lo que se puede modificar y su accesibilidad (por ejemplo, si el juego admite configurar la resolución de pantalla, la calidad de los gráficos, etc.).}
	\end{minipage}}
\end{center}
\section{Monojugador}
\subsection{Descripción}
\begin{center}
	\fbox{\begin{minipage}{32em}
			\textit{El juego, en su vertiente de juego solitario, ¿que le ofrecerá al usuario? ¿Qué objetivos tiene que cumplir el usuario para dar por completado esta parte del juego?}
	\end{minipage}}
\end{center}
\subsection{Modos de juego}
\begin{center}
	\fbox{\begin{minipage}{32em}
			\textit{Detallar los modos de juego que el usuario tendrá disponible. Para cada modo de juego, proporcionar una descripción detallada y si inicialmente es un modo que está bloqueado, incluir instrucciones para que el usuario sepa como tiene que desbloquearlo.}
	\end{minipage}}
\end{center}
\subsection{Horas de gameplay}
\begin{center}
	\fbox{\begin{minipage}{32em}
			\textit{Detallar cuantas horas va a necesitar el jugador para completar al 100\% el juego.}
	\end{minipage}}
\end{center}
\section{Multijugador}
\subsection{Descripción}
\begin{center}
	\fbox{\begin{minipage}{32em}
			\textit{El juego, en su vertiente de juego multijugador, ¿que le ofrecerá al usuario? ¿Qué objetivos tiene que cumplir el usuario para dar por completado esta parte del juego?}
	\end{minipage}}
\end{center}
\subsection{Modos de juego}
\begin{center}
	\fbox{\begin{minipage}{32em}
			\textit{Detallar los modos de juego que el usuario tendrá disponible. Para cada modo de juego, proporcionar una descripción detallada y si inicialmente es un modo que está bloqueado, incluir instrucciones para que el usuario sepa como tiene que desbloquearlo.}
	\end{minipage}}
\end{center}
\subsection{Arquitectura}
\begin{center}
	\fbox{\begin{minipage}{32em}
			\textit{Detallar la arquitectura que se va a seguir para mantener la infraestructura. ¿Arquitectura cliente-servidor o P2P?.}
	\end{minipage}}
\end{center}
\section{Integración con redes sociales}
\begin{center}
	\fbox{\begin{minipage}{32em}
			\textit{Detallar la integración que el juego hace sobre las redes sociales y para qué las utiliza y los beneficios que le pueden traer tanto al jugador como al propio juego.}
	\end{minipage}}
\end{center}
\section{Trucos y Easter Eggs}
\begin{center}
	\fbox{\begin{minipage}{32em}
			\textit{Detallar si se van a incluir trucos y easter eggs dentro del juego. Explicar en detalle en que consiste cada truco y easter egg y especificar que tiene que hacer el usuario para ejecutar el truco o ver el easter egg.}
	\end{minipage}}
\end{center}
